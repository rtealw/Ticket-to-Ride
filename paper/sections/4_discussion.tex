\section{Discussion}
We explain how long routes
are overvalued and can be exploited to easily
win with a very simple strategy.
We then suggest a route scoring scheme that both corresponds
to the resources needed to acquire routes and ensures
that \textit{Ticket to Ride} stays competitive
between several advanced strategies.
We analyze why players
with particular Destination Tickets perform better
than others.
We investigate effective resistance as a measure
for how difficult the cities on Destination Tickets
are to connect and compare resistance (difficulty)
against path length (reward).
The residual of resistance and path length
appears as the best predictor of winning and we provide
a ranking as a tool for choosing
between Destination Tickets in game play.

\subsection{Limitations \& Future Work}
We rely heavily on \textit{Ticket to Ride}
simulations but are limited to four fairly simple
heuristic strategies.
If there exist substantially better strategies,
our scoring recommendation and comparison
of Destination Tickets may be inaccurate.
Future work could explore additional \textit{Ticket to Ride}
and compare them to the four we used.

There are two glaring nuances we ignored:
The first is that collecting Train Car cards
is more complicated than we assume.
In addition to choosing two cards from
some combination of the five face-up cards and deck,
players also have the choice of a single 'wild' card
from the face-up cards. 
This option and the existence of wild
cards complicate the way we discuss turns in terms
of the optimal games with routes of at most length $k$
and the expected cards until $k$ of a particular color
appear.
The second nuance is that some routes can be claimed
with Train Car cards of any color (as long as all Train
Car cards are the same color).
These routes are easier to connect than their counterparts
of the same length and affect our resistance analysis.
Future work could take into account both complications.

Our route value recommendations and
Destination Ticket rankings are specific to the USA map.
Future work may explore how our analysis
applies to non-USA versions such as Europe,
India, and the more than 20 other maps listed on the
Days of Wonder website.
(Simulations from Silva et al. indicate this
is a particularly productive avenue because 
their four heuristic strategies perform radically differently
on other maps \cite{de2017playtesting}.)

We only explore changing the reward for routes.
Future work could analyze changing the reward for
Destination Tickets in conjunction with the
reward for routes.
Such work would likely lead to a more
comprehensive and robust re-calibration
\textit{Ticket to Ride} game play.

\subsection{Acknowledgements}
The authors would like to thank Bill Peterson for
the strategy we used to calculate the expected time
until $k$ blue cards appear in a well-shuffled deck
as described in \cref{sec:collecting_cards}.
