\section{Value of Routes}
\subsection{Current Route Value}
There are two factors that encourage players to buy longer routes:

- the reward per train is much higher for longer routes and

- only one route can be purchased per a turn so it is more efficient
to buy longer routes.

\begin{table}[h!]
\renewcommand{\arraystretch}{1.5}
\centering
\begin{tabular}{| c | c | c | c | c | c | c |}
\hline
 Route Length & 1 & 2 & 3 & 4 & 5 & 6\\
 \hline
 Points Scored & 1 & 2 & 4 & 7 & 10 & 15\\
 \hline
\end{tabular}
\caption{The current values of routes.}
\label{table:current_value}
\end{table}
From \cref{table:current_value} we see that 
the trains in the two shortest routes are each worth only one point
whereas the trains in the longest route are worth two and a half points.
The ostensible reason for the increasing value per train is that the
difficulty of collecting trains grows faster than the number of trains $k$.
The most intuitive way to quantify the difficulty of collecting $k$ trains
is the expected number of turns a player needs to collect them.
However, in \cref{sec:collecting_cards}, 
we show that the expected number of cards
until we get $k$ of a specific color is linear with $k$.

\subsection{Collecting Cards}\label{sec:collecting_cards}
Let $C$ be the set of all cards.
Our goal is to find the expected time until
we have collected $k$ of a specific
color of cards where $k$ is a fixed
integer between 1 and 6 inclusive.
Without loss of generality, say we are
looking for blue cards and
call $B$ the set of blue cards.
In order to find how long it takes to collect $k$ blue cards,
think of our well-shuffled deck as
blue cards separated by non-blue cards.
For example, we may have the ordering
\begin{align}
    xxxbxbxxxx...xxbxxx \nonumber
\end{align}
where $x \in C \setminus B$ and $b \in B$.

Now let $T$ be the number of cards until
the $k^{th}$ blue card is drawn.
Our strategy is to write $T$ in terms of $k$
and indicator random variables.
Let $I_x$ be the indicator that takes value 1
if non-blue card $x$ appears before the $k^{th}$ blue card
and 0 otherwise.
Then
\begin{align}
    T = k + \sum_{x \in C \setminus B} I_x. \nonumber
\end{align}

We are interested in the average number of draws
until we see $k$ blue cards
so we take the expected value of $T$
(distributing expectation over addition),
\begin{align} \label{eq:expected_card}
    \E[T] = \E[k] + (|C \setminus B|) \E[I_x].
\end{align}
We know that $k$ is fixed so its expected value is simply $k$.
Then the only unknown is $\E[I_x]$.

To find $\E[I_x]$ think of the deck as 13 sequences 
of non-blue cards where adjacent sequences are
separated by a single blue card.
(It is possible for a sequence to be of length zero provided
two blue cards are adjacent to each other.)
Since we assume the deck is well-shuffled, the non-blue cards
are uniformly distributed across each sequence;
it is as likely for card $x$ to be in any one sequence
as any other.

Now think of whether a card appears before the $k^{th}$ blue
in terms of which sequence it lives in.
Card $x$ appears before the first blue if and only if it is 
in the first sequence.
The probability $x$ is in the first sequence is $1/13$
so $x$ appears before the first blue with $1/13$ probability.
(Recall that $x$ is uniformly distributed across all
13 sequences.)
Similarly, card $x$ appears before the $k^{th}$ blue if and only if
it is in one of the first $k$ sequences.
The probability $x$ is in the first $k$ sequences is $k/13$
so $x$ appears before the $k^{th}$ blue with $k/13$ probability.
By conditioning on whether $x$ appears before the $k^{th}$ blue,
we may write
\begin{align} \label{eq:expected_indicator}
    \E[I_x] &= 1 \times 
    \p(\textrm{$x$ appears before the $k^{th}$ blue}) +
    0 \times \p(\textrm{$x$ appears after the $k^{th}$ blue})
    \nonumber \\
    &= \p(\textrm{$x$ appears before the $k^{th}$ blue})
    = k/13.
\end{align}

With \cref{eq:expected_card} and \cref{eq:expected_indicator},
\begin{align}
    \E[T] &= k + (|C \setminus B|) k/13 \nonumber \\
    &= k \left(1 + \frac{|C| - |B|}{13}\right). \nonumber
\end{align}

\subsection{Converting Cards to Turns}\label{sec:converting}