\section{Value of Routes}
\subsection{Current Route Value}
There are two factors that encourage players to buy longer routes:

- the reward per train is much higher for longer routes and

- only one route can be purchased per a turn so it is more efficient
to buy longer routes.

From \cref{table:current_value} we see that 
the trains in the two shortest routes are each worth only one point
whereas the trains in the longest route are worth two and a half points.
The ostensible reason for the increasing value per train is that the
difficulty of collecting trains grows faster than the number of trains $k$.
The most intuitive way to quantify the difficulty of collecting $k$ trains
is the expected number of turns a player needs to collect them.
However, in \cref{sec:collecting_cards}, 
we show that the expected number of cards
until we get $k$ of a specific color is linear with $k$.

\begin{table}[h!]
\renewcommand{\arraystretch}{1.5}
\centering
\begin{tabular}{| c | c | c | c | c | c | c |}
\hline
 Route Length & 1 & 2 & 3 & 4 & 5 & 6\\
 \hline
 Points Scored & 1 & 2 & 4 & 7 & 10 & 15\\
 \hline
 Train Value & 1 & 1 & $1.\overline{3}$ & 1.75 & 2 & 2.5\\
 \hline
\end{tabular}
\caption{The point reward for building routes
of varying lengths.}
\label{table:current_value}
\end{table}

\subsection{Collecting Cards}\label{sec:collecting_cards}
In this section, we find the expected number of cards
$N$ we need to see in order to find the $k$ cards
to build a particular route.
Our argument is that $\E[N]$ is a better than the
current reward scheme.

Let $C$ be the set of all cards and $k$ be a fixed integer
between 1 and 6, inclusive.
Without loss of generality say we are looking
for blue cards and call this set $B$.
In order to find how long it takes to find $k$ blue cards,
think of our well-shuffled deck as
blue cards separated by non-blue cards.
For example, our deck may have the ordering
\begin{align}
    xxxbxbxxxx...xxbxxx \nonumber
\end{align}
where $x \in C \setminus B$ and $b \in B$.

Now let $N_k$ be the number of cards until
we see the $k^{th}$ blue card.
Our strategy is to write $N_k$ in terms of
indicator random variables.
Let $I_{k,x}$ be the indicator that takes value 1
if non-blue card $x$ appears before the $k^{th}$ blue card
and 0 otherwise.

The number of cards until the $k^{th}$ blue card
is certainly the number of blue cards $k$
plus the number of non-blue cards before the $k^{th}$ blue.
Written as an equation,
\begin{align}
    N_k = k + \sum_{x \in C \setminus B} I_{k,x}. \nonumber
\end{align}
We are interested in the average number of cards
so we take the expectation of $N_k$ and distribute
over addition via linearity of expectation.
Then
\begin{align} \label{eq:expected_card}
    \E[N_k] = \E[k] + |C \setminus B| \times \E[I_{k,x}].
\end{align}
Since $k$ is fixed, $\E[k]$ is simply $k$.
Since $B$ is a subset of $C$ and both sets are fixed,
$\E[|C \setminus B|]$ is $|C| - |B|$.
Thus we need only find $\E[I_{k,x}]$.

To calculate $\E[I_{k,x}]$, think of the deck as $|B| + 1$ sequences 
of non-blue cards separated by $|B|$ blue cards.
(It is possible for a sequence to be of length zero
in the case that two blue cards are adjacent to each other or
in the case that the first or last card is blue.)
Since we assume the deck is well-shuffled, the non-blue cards
are uniformly distributed across the $|B| + 1$ sequences:
it is as likely for card $x \in C \setminus B$ to be in any one sequence
as any other.

Now think of the indicator $I_{k,x}$ that card $x$ appears
before the $k^{th}$ blue in terms of which of the $|B| + 1$
sequences $x$ resides in.
When looking for only one blue card,
we see that card $x$ appears before the first blue if and only if
$x$ is in the first sequence.
Since $x$ is uniformly distributed across all $|B| + 1$ sequences,
the probability that $x$ appears before the first blue $\p(I_{1,x}$
is $1/(|B| + 1)$.
Similarly, the probability that $x$ appears before the second blue
$\p(I_{2,x}$ is $2/(|B| + 1)$.
By extension, the probability that $x$ appears before the $k^{th}$
$\p(I_{k,x}$ is $k/(|B| + 1)$.

Recall that $I_{k,x}$ takes value 1 if $x$ appears before the 
$k^{th}$ blue and 0 otherwise.
Then, conditioning on $I_{k,x}$, we write its expectation
\begin{align} \label{eq:expected_indicator}
    \E[I_{k,x}] &= 1 \times \p(I_{k,x})
    + 0 \times (1-\p(I_{k,x}) \nonumber \\
    &=\p(I_{k,x}) = \frac{k}{|B| + 1}
\end{align}
Substituting \cref{eq:expected_indicator} in \cref{eq:expected_card},
we have
\begin{align}
    \E[N_k] &= k + |C \setminus B| \times \left(\frac{k}{|B| + 1}\right)
    \nonumber \\
    &= k \left(1 + \frac{|C| - |B|}{|B| + 1} \right). \nonumber
\end{align}
and, by plugging in the values of $|C|$ and $|B|$,
\begin{align}
    \E[N_k] &= k \left(1 + \frac{110 - 12}{12 + 1} \right) \nonumber \\
    &= k \left(\frac{111}{13}\right) \approx 8.538k
\end{align}

\subsection{Converting Cards to Turns}\label{sec:converting}